\chapter{Arhitecturi de conectare  MPTCP prin proxy}
\label{sec:arch_upb}

MPTCP RFC\cite{rfc6824bis}, \cite{mptcp-nsdi} este un upgrade la TCP
care permite conectarea simultană prin interfețe multiple, fără a
recompila aplicațiile. O aplicație folosește în continuare API-ul de
BSD sockets, dar beneficiază de conexiunile multiple disponibile în
telefoanele de astăzi: WiFi, 4G, si Bluettoth. Pentru a beneficia de
funcționalitatea MPTCP, atât serverul cât si clientul trebuie
modificate. Deoarece multe servere nu implementează încă MPTCP,
soluția temporara este de a ruta tot traficul MPTCP printr-un proxy
care continuă conexiunea cu TCP clasic până la serverele legacy.


\section{Protocoale din familia SOCKS}

\subsection{SOCKS 5}

Protocolul SOCKS este folosit pentru a crea conexiuni TCP către
destinații arbitrare folosind un proxy.  În ultima versiune
standardizată a protocolului (versiunea 5), durează doua RTT-uri (sau
3, dacă se face și autentificare) până când datele pot circula între
client și server.

\begin{figure}[h]
	\centering
	\includegraphics[scale=0.7]{figures/socks/socks5op.png}
	\caption{Mod de operare SOCKS 5}
    	\label{fig:socks5op}
\end{figure}

In SOCKS 5 (versiunea curentă a protocolului, v. fig
\ref{fig:socks5op}), clientul deschide o conexiune căte proxy si
parcurge urmatoarele etape înainte să transmită date către server:
\begin{itemize}
	\item Negocierea metodei de autentificare (1 RTT): Clientul trimite un mesaj care conține metodele de autentificare suportate. Proxy-ul răspunde cu metoda de autentificare aleasă.
	\item Autentificarea propriu-zisă (0-1 RTT-uri): Acest pas poate să lipsească, dacă nu se face autentificare. Altfel, durează tipic un RTT.
	\item Crearea socket-ului (1 RTT): Clientul trimite adresa și portul server-ului la care vrea sa se conecteze. Proxy-ul încearcă să onoreze cererea clientului și îi trimite un răspuns cu rezultatul.
\end{itemize}


\subsection{SOCKS 6}

Am dezvoltat versiunea 6 a protocolului SOCKS, pe care o propunem
pentru standardizare în cadrul IETF. Momentan se află in starea de
Internet Draft~\cite{socks6}.  Principalele îmbunătățiri introduse în
această versiune sunt:
\begin{itemize}
	\item Clientul are un comportament optimist și trimite cât mai multe informații către proxy, fără a aștepta să se termine autentificarea.
	\item Semanticile cererilor imită semanticile TCP Fast Open~\cite{rfc7413}. În cerere, clientul poate include și potențialul payload pentru SYN-ul inițial trimis către server.
	\item Protocolul poate fi extins folosind opțiuni, similare cu opțiunile TCP.
	\item Folosint opțiunile menționate mai sus, se pot implementa scheme de autentificare în 0 RTT-uri.
\end{itemize}


\begin{figure}[h]
	\centering
	\includegraphics[scale=0.7]{figures/socks/socks6op1st.png}
	\caption{Mod de operare SOCKS 6 (prima conexiune cu autentificare)}
    	\label{fig:socks6op1st}
\end{figure}

\begin{figure}[h]
	\centering
	\includegraphics[scale=0.7]{figures/socks/socks6op2nd.png}
	\caption{Mod de operare SOCKS 6 (conexiuni ulterioare)}
    	\label{fig:socks6op2nd}
\end{figure}

Atunci când un client SOCKS 6 încearcă să se conecteze la un server, deschide o conexiune către un proxy (v. fig. \ref{fig:socks6op1st}) .
Clientul începe prin a trimite o cerere (SOCKS Request), care conține, printre altele:
\begin{itemize}
	\item Metodele de autentificare cunoscute de client.
	\item Adresa și portul server-ului.
	\item Primii octeți din fluxul de date destinat server-ului.
\end{itemize}

Proxy-ul răspunde cu un Authentication reply, care indică protocolul de autentificare care trebuie folosit.
Apoi se execută protocolul de autentificare ales, lucru care poate dura un RTT sau mai mult.

Dacă autentificarea s-a terminat cu succes, proxy-ul încearcă să creeze conexiunea cerută de client, si trimite un Operation Reply, care indică dacă s-a putut realiza conexiunea sau nu.

În cererile ulterioare, clientul poate să includă date de autentificare in cerere (Request), lucru care îl scutește de etapa de autentificare (fig. \ref{fig:socks6op1st}).
Astfel, se poate obține un răspuns de la server intr-un singur RTT, cu 2-3 RTT-uri mai rapid decât cu SOCKS 5.


Implementarea valabilă la momentul depunerii primului draft la IETF
este disponibilă online la \cite{socks105}. 



\chapter{Folosirea MPTCP în sistemul de operare Android}
\label{sec:mptcp_android}
\section{Imaginea Android}

În acest proiect s-au folosit dispozitive mobile Samsung Galaxy S7 Edge. Acestea rulează în mod implicit o imagine a sistemului de operare Android de la Samsung, care include implementarea protocolului MPTCP în kernel. 

Imaginea de Android (stock ROM) folosită este identificată prin urmatoarele:
\begin{enumerate}
	\item Model number: SM-G935F
	\item Baseband version: G935FXXU1DQA3
	\item Build number: NRD90M.G935FXXU1DQAS
	\item Android version: 7.0
\end{enumerate}

Imaginea de Android stock este rescrisă folosind utilitarul Odin, în timp ce dispozitivul este în modul \texttt{Download}. Astfel, se va rescrie partițiile boot, system și recovery ale dispozitivului. Prin această metodă, telefonul este adus înapoi la starea inițială, în cazul întâlnirii unei erori pe parcursul root-ării dispozitivului. 

TODO

\section{Rootarea dispozitivului mobil}

Pentru activarea și configurarea protocolului MPTCP este nevoie de un telefon rootat deoarece comenzile necesare nu pot fi rulate decat din contul root (sysctl, ip rule, ip route).

Pentru root-area dispozitivului este nevoie de utilitarele Odin, TWRP și SuperSU. Se foloseste imaginea de recovery TWRP pentru modelul \emph{hero2lte}. Se scrie această imagine folosind Odin, în timp ce telefonul este în modul \texttt{Download}. Astfel, se va rescrie partiția de recovery a dispozitivului. 

Apoi se intră în modul \texttt{Recovery} al telefonului pentru a accesa meniul aplicației TWRP. Se formatează partiția de date și se instalează aplicația SuperSU care rootează telefonul. În final se boot-ează telefonul în modul normal. Apoi, în adb shell se poate introduce comanda \texttt{su} pentru a intra în contul utilizatorului root.

TODO

\section{Configurarea protocolului MPTCP}

Configurarea protocolului MPTCP se face prin următorii pași:
\begin{enumerate}
	\item Se activează WiFi și LTE în Settings
	\item Se activează opțiunea Settings $->$ Developer Options $->$ Mobile Data Always Active
	\item Se activează protocolul MPTCP folosind utilitarul \texttt{sysctl}
	\item Se crează câte un tabel de rutare diferit pentru fiecare interfață, care vor fi folosite pentru rutarea pe baza adresei sursă, prin comanda \texttt{ip rule}
	\item Se configurează cele două tabele de rutare prin adaugarea unei rute default, folosind gateway-ul fiecărei conexiuni, prin comanda \texttt{ip route} 
	\item Se adaugă o rută default globală pentru traficul normal în Internet, folosind gateway-ul uneia din cele două conexiuni, prin comanda \texttt{ip route}
\end{enumerate}
Pașii 3-6 se pot automatiza sub forma unui script bash, care se poate rula după boot-area dispozitivului sau la cererea utilizatorului, atunci când dorește activarea MPTCP.  Se va crea și configura cate un tabel de rutare pentru fiecare interfață activă (cu adresă IP).

De asemenea, la dezactivarea unei interfețe sau la pierderea conexiunii pe o interfață, trebuie șteasă tabela de rutare asociată cu acea interfață, prin comanda \texttt{ip rule}. 

TODO

\section{Aplicația de monitorizare a conexiunilor de rețea}

A fost dezvoltată o aplicație Android, numită ConnectivityMonitor,  care monitorizează starea conexiunilor WiFi și LTE, și reconfigurează tabelele de rutare pentru MPTCP atunci cand se modifică starea unei conexiuni. De asemenea, poate fi folosită pentru vizualizarea statisticilor colectate sau pentru rularea scripturilor/executabilelor. 

Funcționalitatea de bază a aplicației este de a activa și configura MPTCP pentru folosirea celor două interfețe de rețea (WiFi și LTE). Utilizatorul poate activa si dezactiva MPTCP folosind o opțiune din fereastra de setări. În spate, aplicația rulează scripturi bash în mod privilegiat (folosind utilitarul \texttt{su}).

Aplicația este implementată să primească notificări atunci când se schimbă starea unei interfețe de rețea (conectare și deconectare). In cazul unui eveniment de conectare, se crează și populează tabela de rutare asociată interfeței. În cazul unui eveniment de deconectare, se șterge tabela de rutare asociată acelei interfețe.

ConnectivityMonitor colectează informații despre cele doua interfețe, periodic, o data la 30 secunde. Pentru interfața de WiFi se colectează: numărul de bytes trimiși și primiți, RSSI, RTT, MCS și frecvența. Pentru interfața de LTE se colectează: numărul de bytes trimiși și primiți, RSSI, CID, TAC. De asemenea, este salvat periodic nivelul bateriei. 

Tot din aplicație se pot rula scripturi bash, Python sau executabile, pentru evaluarea experimentală a caracteristicilor traficului de rețea (lățime de bandă, lantență, etc.).

Evenimentele de conectare/deconectare a interfețelor de rețea, datele colectate și rezultatele scripturilor sunt salvate in baza de date, care este salvată periodic în cloud, în spațiul de stocare oferit de Firebase. 

Informațiile din baza de date pot fi analizate și corelate pentru determinarea procentului de timp în care utilizatorul are acces la ambele interfețe de rețea, disponibilitatea lor atunci când utilizatorul dorește să le utilizeze, și perfomanța acestora în  acele momente.




