\chapter{Obiective}
\label{sec:objectives}

\section{Obiective etapa 2017}

\begin{itemize}
\item Implementarea modificărilor necesare pe telefoane: modificarea
kernel-ului și actualizări la Android pentru a folosi
MPTCP. Sistemul Android se bazează pe Linux, dar există suficiente
diferențe și părți specifice dispozitivului pentru a face un port
compatibil cu toate aplicațiile non-trivial.

\item Implementare și plasare proxy în rețeua Orange România
SGiLAN. Proxy-ul poate fi explicit sau transparent, fiecare soluție necesită
o implementare diferită pe telefon și la proxy.


\item Executare măsurători de performanță pentru diferite scenarii
  pentru a optimiza: conectivitatea, capacitatea, experiența cât mai
  fluidă a utilizatorului. Deoarece actualizarea MPTCP este invazivă
  pentru stiva TCP/IP, nu trebuie să existe nicio regresie a
  funcționalității sau performanțelor atunci când serviciul este
  implementat într-o rețea de testare cu o utilizatori reali.


\item Realizarea unui prototip pentru testarea la scară mai largă,
  folosind echipe de testare ale operatorului sau grupuri de
  utilizatori specializați.


\end{itemize}
