
\chapter{Introducere}

Atunci cand a fost creata, arhitectura Internet-ului a fost bazata pe
principiul end-to-end, prin care se limiteza rolul retelei la a
transmite pachete, iar orice prelucrare la nivel superior in stiva de
protocoale este executata de sistemele de la capetele
conexiunilor. Acest model are marea virtute de a permite noi
aplicatii, necesitand numai modificarea calculatoarelor de la capete,
nu si a retelei.

Se poate argumenta ca internetul nu face nimic extraordinar de bine,
dar face aproape totul destul de bine, si aceasta este ceea ce
conteaza. Din pacate, realitatea nu este tocmai asa: Internet-ul nu
face totul suficient de bine.  Versiunea initiala a arhitecturii
Internet a incetat sa descrie Internet-ul in practica de aproximativ
cincisprezece ani.  NAT-urile si firewall-urile au fost primele
dispozitive folosite in practica, in retea, care proceseaza pachetele
la nivel 4 (sau mai sus).  Mai recent, a devenit o practica larg
raspandita folosirea inpectiei profunde a pachetelor (deep packet
inspection, sau DPI) pentru a analiza traficul si a limita largimea de
banda pentru traficul ``mai putin important''.  Proxy-uri transparente
web si acceleratoare de aplicatii sunt montate in retea pentru a
imbunatati performanta, sisteme de detectare a intruziunilor
reasambleaza fluxurile pentru a monitoriza traficul, iar
normalizatoarele de trafic elimina cazuri ambigue care ar putea
reprezenta amenintari pentru sistemele de securitate. Numim toate
aceste dispozitive de retea care proceseaza traficul la nivel 4 sau
mai sus ``middleboxes''.

Aceasta serie de imbunatatiri ad-hoc ale arhitecturii initiale aduce
doua probleme-cheie. In primul rand, ele incorporeaza limitele
actualelor protocoale si aplicatii in cadrul retelei, facand
functionalitatile noi foarte greu de implementat.  In al doilea rand,
ele fac dificila sarcina de a intelege ce se va intampla cu un pachet
trimis: iar aceasta face Internet-ul fragil si greu de depanat.

Am realizat recent un studiu care arata ca middleboxurile sunt
predominante astazi (detalii puteti gasi in \cite{imc}). Am testat 142
retele de acces in 24 de tari. 33\% din caile testate contin un
middlebox care pastreaza stare despre fluxuri in retea si
implementeaza functionalitate de nivel 4+ in retea. Aceste cifre nu se
iau in considerare si middlebox-urile care implementeaza NAT si
firewalls; acestea sunt omniprezente. De asemenea, nu poate observa
dispozitive de procesare in retea la servere, cum ar fi cele care
balanseaza traficul catre serverele dintru centru de date (data
center).

Observatie este ca Internetul originar, transparent end-to-end, a
evoluat intr-o retea in care o cale end-to-end este compusa din mai
multe segmente de retea IP interconectate de puncte care proceseaza
trafic la nivel 4 sau mai sus.  Pare inutil sa incercam de a combate
aceasta tendinta: pur si simplu, exista prea multe motive pentru care
operatorii aleg sa implementeze aceste middleboxes. Singura strategie
rezonabila este de a imbratisa middleboxes, propunand insa o
arhitectura care sa le includa, o arhitectura care pe de o parte sa
permita clientilor Internet sa inteleaga functionalitatea pe care ele
o ofera, si care permit deasemenea clientilor sa instantieze
functionalitate noua in retea.

Daca ne-am limita la a face middlebox-urile vizibile, castigurile ar
fi limitate. Ceea ce dorim sa facem cu adevarat este sa construim o
retea in care aplicatii noi pot folosi de middleboxes pentru la
imbunatati experienta utilizatorilor, oferind in stimulente
operatorilor de Internet de a rula aplicatii pentru clienti in
mijlocul retelei.

Este posibil sa cream o arhitectura in care middlebox-urile sa
mareasca abilitatea noastra de a instala aplicatii noi in retea,
permitand astfel Internetul-ui sa continua sa evolueze?


%%%%%%%%%%%%%%%%%%%%%%%%%%%%%%%%%%%%%%%%%%%%%%%%%%%%%%%%%%%%%%%%%%%%%%
%%% introduction.tex ends here
