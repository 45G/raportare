\chapter{Arhitectura SOCKS v6}
\label{sec:arch_upb}

% se reia cadrul general cu desenele pentru SOCKSv6

Operatorii de telefonie mobila instaleaza MPTCP \cite{Raiciu_Hard:2012} pe telefoane mobile 
pentru a exploata latimea de banda oferita de LTE si WiFi in acelasi timp. Deoarece foarte putine
servere folosesc MPTCP, este necesar un proxy in incinta operatorului, care sa termine conmexiunea
MPTCP si sa comunice cu serverul folosind TCP normal.

%TODO

\chapter{Caracteristici SOCKS v6}
% spunem despre ce-și doresc mobilele, MPTCP-ul, și securitatea.
% în principiu, ce s-a adăugat în draft-01 șî draft-02
\label{sec:arch_upb}

Versiunile 4 si 5 \cite{rfc1928} ale protocolului SOCKS au fost dezvoltate si standardizate acum
doua decenii si sunt larg folosite in ziua de azi pentru diverse scopuri.
Eel sunt implementate in multe aplicatii, precum browsere web, clienti SSH si proxificator.

Este foarte important ca un proxy sa ofere performanta buna si sa profite de cele mai recente
imbunatatiri aduse protocoalelor de retea. Pentru a profita de de TCP Fast Open \cite{rfc7413}
sau pentru oferi securitate mai buna impotriva potentialilor atacatori, Korea Telecom si alti utilizatori
au recurs la proxy-uri precum Shadowsocks \cite{shadowsocks}, care folosesc protocoale nestandardizate
si a caror securitate nu au fost auditata.

Acest capitol descrie protocolul SOCKS versiunea 6, care acum este in curs de dezvoltate la IETF \cite{olteanu-intarea-socks-6-02}.
Principala imbunatatire adusa protocolului este eliminarea round-trip-urilor de date dintre client si proxy in cele mai multe cazuri.
SOCKS versiunea 6 rezolva problemele principale ale forlosirii atat TFO, cat si a 0-RTT TLS Session Resumption \cite{tls1.3},
oferind un mecanism care asigura idempotenta cererilor.

SSOCKS v6 este de asemenea extensibil, permitand implementarea unor functionaltiati noi, fara a introduce probleme de compatibilitate.

\section{RTT redus}


\begin{figure*}[t]
	\centering
	\includegraphics*[angle=0,height=.50\textwidth]{figures/socks/a-crop}
	\label{fig:socksv5}
	\caption{SOCKS v5}
\end{figure*}

\begin{figure*}[t]
	\centering
	\includegraphics*[angle=0,height=.50\textwidth]{figures/socks/b-crop}
	\label{fig:socksv6short}
	\caption{SOCKS v6}
\end{figure*}

\begin{figure*}[t]
	\centering
	\includegraphics*[angle=0,height=.50\textwidth]{figures/socks/c-crop}
	\label{fig:yes_yes_no}
	\caption{SOCKS v6 cu TFO disponibil numai intre client si proxy}
\end{figure*}

Atunci cand un client doreste se se conecteze prin proxy la un server, trebuie sa deschida o conexiune TCP
catre un proxy SOCKS v6. Diferenta principala fata de versiunea 5 este ca se transmit cat mai multe informatii
inca din primul mesaj. Primul mesaj trimis de client se numeste un SOCKS Request si  contine adresa si
portul serverului, alaturi de o transa initiala de date pentru server (de exemplu, un HTTP GET).
Daca clientul si proxy-ul au comunicat in prealabil, este posibila autentificarea in 0 RTT-uri, caz in care
datele de autentificare sunt si ele incluse in cerere.

Apoi, proxy-ul trimite un Authentication Reply. Daca in Request nu se aflau date de autentificare satisfacatoare,
proxy-ul indica metoda de autentificare ce trebuie sa urmeze. Acest mesaj este urmat de protocolul de autentificare 
specific metodei.

In final, proxy-ul se conecteaza la server si genereaza un Operation Reply, prin indica succesul sau esecul conexiunii.
Datele primite ulterior de proxy sunt trimite verbatim catre client sau server.

In cazul cel mai uzual, in care autentificarea este bine pusa la punct (v. fig. \ref{fig:socksv6short}), proxy-ul
incearca sa se conecteze la server imediat ce promeste Request-ul, neadaugand latenta in plus.

Presupunand ca proxy-ul se afla pe calea dintre client si server, timpul necesar pentru a obtine un raspuns de la server
(de ex. un HTTP OK) prin intermediul SOCKS v6 nu este mai lung decat timpul obtinerii unui raspuns atunci cand serverul
este contactat direct.

\begin{table}
	\centering
	\begin{tabular}{| l | c | c | c | r |} \hline
		& TFO la proxy & TFO la server & RTT total \\ \hline
	% 	SOCKSv5 & \multicolumn{3}{|l|}{~}  & ? 3T + 3t ? \\ \hline
		\multirow{2}{*}{TCP} & - & Nu  & \(2P+2S\) \\ \hhline{~----}
		~ & - & Da  & \(P+S\) \\ \hline
		\multirow{3}{*}{SOCKSv6} & Nu & Nu  & \(2P + 2S\)  \\ \hhline{~----}
		~ & Da & Nu & \(P + 2S\) \\ \hhline{~----}
		~ & Da  & Da & \(P + S\) \\ \hline
	\end{tabular}
  	\caption{Timpul necesar primirii unui raspuns de date. (P = RTT client-proxy, S = RTT proxy-server. P + S = RTT client-server.)}
 \end{table}

SOCKS v6 este mai rapid decat TCP atunci cand serverul nu are TFO activat.
Clientii pot folosi TFO pe traseul dintre proxy si server, eliminand un RTT dintre client si proxy.
Acest lucru este foarte avantajos pentru dispozitivele mobile, intrucat latenta dintre dispozitiv si base station
poate fi foarte mare.

In plus, daca rulam SOCKS v6 peste TLS, timpii obtinuti nu se schimba atata timp cat folosim 0-RTT Session Resumption.

Spre diferenta de versiunea 6, in versiunea 5 sunt necesare 2 RTT-uri (sau 3, daca e necesara autentificarea clientilor)
pana cand datele incep sa curga intre client si server.

\section{Operațiuni idempotente cu TLS 1.3}

Atat TCP Fast Open, cat si 0-RTT TLS Session Resumption au probleme bine cunoscute care pot duce la duplicarea unor
cereri din partea clientului.

In cazuri rare, retransmiterea unui SYN cu TFO si date in payload poate duce la creearea unei conexiuni duplicate la server.
Acest lucru poate fi dezastruos daca, de exemplu, serverul se ocupa de tranzactii financiare si proceseaza o tranzactie de doua ori.

In TLS 1.3, un client care deschide o conexiune noua si doreste sa-si continue sesiunea de TLS poate sa includa niste date (de ex. un HTTP GET)
in primul mesaj trimis catre server. Aceste date initiale (numite "Early Data") beneficiaza de securitate mai scazuta.
Daca intre cele doua capete se afla un atacator, acesta poate sa captureze datele initiale (criptate) si sa creeze conexiuni duplicate.

Pentru a folosi in deplina siguranta TFO si 0-RTT Session Resumption intre client si proxy, am implementat un mecanism
prin care se asigura idempotenta cererilor SOCKS. Proxy-ul va onora o cerere individuala cel mult o data, indiferent de cate
ori o primeste.

Pentru a se proteja de cereri duplicate, clientii pot cere, si apoi cheltui, token-uri de idempotenta. Un token poate fi cheltuit pe o singura cerere.

Token-ii sunt intregi fara semn de 4 octeti intr-un spatiu modular de 4 octeti. Astfel, daca \(x\) si \(y\) sunt tokeni, \(x < y\) daca \( 0 < (y - x) < 2^{31} \) (in aritmetica pe 32 de biti).

Proxy-ul acorda intervale contigue de token-uri numite \emph{ferestre}. Aceste ferestre sunt definite de baza (primul token din interval) si de dimensiune.
Ferestrele sunt mutate (adica li se incrementeaza baza, dar isi pastreaza dimensiunea) de catre proxy.

Clientii incearca sa isi cheltuiasca token-urile in ordine. Acest lucru nu garanteaza ca proxy-ul va primi sau va procesa cererile in ordine. De asemenea, unele cereri se pot pierde din cauza retelei.

Proxy-ul urmareste starea tuturor token-ilor din fereastra, mutand fereasta pe masura ce token-urile de la inceputul ferestrei sunt cheltuiti.
Pentru ca un token necheltuit sa nu opreasca avansarea ferestrei, am introdus un parametru numit \emph{high water mark}. Acest parametru
este un numar strict mai mic decat dimensiunea ferestrei. Daca diferenta dintre cel mai mare token cheltuit si cel mai mic token necheltuit este mai
mare decat high water mark, fereastra este mutata si token-ul necheltuit este pierdut. Cresterea acestui parametru face proxy-ul mai tolerant la cereri reordonate.

Mecanismul de idempotenta are urmatoarele proprietati:
\begin{itemize}
    \item \emph{Consum mic de resurse}: Consumul de memorie pe client este fix, nu se fac realocari de memorie si toate oreratiunile sunt ieftine.
    Mecanismul de idempotenta in sine nu ofera o posibilitate de denial of service.
    \item \emph{Tolerant la crash-uri}: Repornirea proxy-ului va duce la alocarea unei alte ferestre. Din moment ce este foarte improbabil ca
    fereastra noua sa se intersecteze cu chea veche, sansele de a onora o cerere veche sunt foarte mici.
    \item \emph{Sigur}: Pentru a duplica o cerere cu succes, un atacator trebuie sa astepte pana cand fereastra parcurge tot spatiul modular de numere si ajunge inapoi in pozitia initiala
    (adica pana cand clientul face 4 miliarde de cereri).
    \item {Nerestrictiv}: Singura restrictie impusa clientului esre ca poate face un numar limitat de cereri intr-un RTT. Acest lucru nu e o problema daca ferestrele sunt suficient de mari.
\end{itemize}


% 
% Our chosen solution has the following properties:
% \emph{a) Light weight}: The memory usage is constant,
% the proxy performs only one memory allocation per client, and all operations are cheap.
% This mechanism does not open up an avenue for DoS by itself.
% \emph{b) Resilient to proxy crashes}: Restarting the proxy would ultimately result in a new token window being allocated.
% Since the new window is very unlikely to overlap with the old one, the odds of honoring an old request are very low.
% \emph{c) Resilient across time}: To successfully replay a request, an attacker would have to wait until the window
% shifts back to the same position (i. e until the client makes 4 billion requests).
% \emph{d) Timely}: Token window advertisements are sent immediately after receiving the request, as part of the authentication reply.
% \emph{e) Unrestrictive}: The only restriction placed on the client is that it can make a limited number of requests per client-proxy RTT,
% which is not an issue if the window size is large enough.
% Token expenditure requests and window advertisements can be received out-of order.

\section{Opțiuni de configurare a sockeților pe proxy}

