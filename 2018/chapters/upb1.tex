\chapter{Arhitectura SOCKS v6}
\label{sec:arch_upb}

% se reia cadrul general cu desenele pentru SOCKSv6

Operatorii de telefonie mobila instaleaza MPTCP \cite{Raiciu_Hard:2012} pe telefoane mobile 
pentru a exploata latimea de banda oferita de LTE si WiFi in acelasi timp. Deoarece foarte putine
servere folosesc MPTCP, este necesar un proxy in incinta operatorului, care sa termine conmexiunea
MPTCP si sa comunice cu serverul folosind TCP normal.

%TODO

\chapter{Caracteristici SOCKS v6}
% spunem despre ce-și doresc mobilele, MPTCP-ul, și securitatea.
% în principiu, ce s-a adăugat în draft-01 șî draft-02
\label{sec:arch_upb}

Versiunile 4 si 5 \cite{rfc1928} ale protocolului SOCKS au fost dezvoltate si standardizate acum
doua decenii si sunt larg folosite in ziua de azi pentru diverse scopuri.
Eel sunt implementate in multe aplicatii, precum browsere web, clienti SSH si proxificator.

Este foarte important ca un proxy sa ofere performanta buna si sa profite de cele mai recente
imbunatatiri aduse protocoalelor de retea. Pentru a profita de de TCP Fast Open \cite{rfc7413}
sau pentru oferi securitate mai buna impotriva potentialilor atacatori, Korea Telecom si alti utilizatori
au recurs la proxy-uri precum Shadowsocks \cite{shadowsocks}, care folosesc protocoale nestandardizate
si a caror securitate nu au fost auditata.

Acest capitol descrie protocolul SOCKS versiunea 6, care acum este in curs de dezvoltate la IETF \cite{olteanu-intarea-socks-6-02}.
Principala imbunatatire adusa protocolului este eliminarea round-trip-urilor de date dintre client si proxy in cele mai multe cazuri.
SOCKS versiunea 6 rezolva problemele principale ale forlosirii atat TFO, cat si a 0-RTT TLS Session Resumption \cite{tls1.3},
oferind un mecanism care asigura idempotenta cererilor.

SSOCKS v6 este de asemenea extensibil, permitand implementarea unor functionaltiati noi, fara a introduce probleme de compatibilitate.

\section{RTT redus}


\begin{figure*}[t]
	\centering
	\includegraphics*[angle=0,height=.50\textwidth]{figures/socks/a-crop}
	\label{fig:socksv5}
	\caption{SOCKS v5}
\end{figure*}

\begin{figure*}[t]
	\centering
	\includegraphics*[angle=0,height=.50\textwidth]{figures/socks/b-crop}
	\label{fig:socksv6short}
	\caption{SOCKS v6}
\end{figure*}

\begin{figure*}[t]
	\centering
	\includegraphics*[angle=0,height=.50\textwidth]{figures/socks/c-crop}
	\label{fig:yes_yes_no}
	\caption{SOCKS v6 cu TFO disponibil numai intre client si proxy}
\end{figure*}

Atunci cand un client doreste se se conecteze prin proxy la un server, trebuie sa deschida o conexiune TCP
catre un proxy SOCKS v6. Diferenta principala fata de versiunea 5 este ca se transmit cat mai multe informatii
inca din primul mesaj. Primul mesaj trimis de client se numeste un SOCKS Request si  contine adresa si
portul serverului, alaturi de o transa initiala de date pentru server (de exemplu, un HTTP GET).
Daca clientul si proxy-ul au comunicat in prealabil, este posibila autentificarea in 0 RTT-uri, caz in care
datele de autentificare sunt si ele incluse in cerere.

Apoi, proxy-ul trimite un Authentication Reply. Daca in Request nu se aflau date de autentificare satisfacatoare,
proxy-ul indica metoda de autentificare ce trebuie sa urmeze. Acest mesaj este urmat de protocolul de autentificare 
specific metodei.

In final, proxy-ul se conecteaza la server si genereaza un Operation Reply, prin indica succesul sau esecul conexiunii.
Datele primite ulterior de proxy sunt trimite verbatim catre client sau server.

In cazul cel mai uzual, in care autentificarea este bine pusa la punct (v. fig. \ref{fig:socksv6short}), proxy-ul
incearca sa se conecteze la server imediat ce promeste Request-ul, neadaugand latenta in plus.

Presupunand ca proxy-ul se afla pe calea dintre client si server, timpul necesar pentru a obtine un raspuns de la server
(de ex. un HTTP OK) prin intermediul SOCKS v6 nu este mai lung decat timpul obtinerii unui raspuns atunci cand serverul
este contactat direct.

\begin{table}
	\centering
	\begin{tabular}{| l | c | c | c | r |} \hline
		& TFO la proxy & TFO la server & RTT total \\ \hline
	% 	SOCKSv5 & \multicolumn{3}{|l|}{~}  & ? 3T + 3t ? \\ \hline
		\multirow{2}{*}{TCP} & - & Nu  & \(2P+2S\) \\ \hhline{~----}
		~ & - & Da  & \(P+S\) \\ \hline
		\multirow{3}{*}{SOCKSv6} & Nu & Nu  & \(2P + 2S\)  \\ \hhline{~----}
		~ & Da & Nu & \(P + 2S\) \\ \hhline{~----}
		~ & Da  & Da & \(P + S\) \\ \hline
	\end{tabular}
  	\caption{Timpul necesar primirii unui raspuns de date. (P = RTT client-proxy, S = RTT proxy-server. P + S = RTT client-server.)}
 \end{table}

SOCKS v6 este mai rapid decat TCP atunci cand serverul nu are TFO activat.
Clientii pot folosi TFO pe traseul dintre proxy si server, eliminand un RTT dintre client si proxy.
Acest lucru este foarte avantajos pentru dispozitivele mobile, intrucat latenta dintre dispozitiv si base station
poate fi foarte mare.

In plus, daca rulam SOCKS v6 peste TLS, timpii obtinuti nu se schimba atata timp cat folosim 0-RTT Session Resumption.

Spre diferenta de versiunea 6, in versiunea 5 sunt necesare 2 RTT-uri (sau 3, daca e necesara autentificarea clientilor)
pana cand datele incep sa curga intre client si server.

\section{Operațiuni idempotente cu TLS 1.3}

Atat TCP Fast Open, cat si 0-RTT TLS Session Resumption au probleme bine cunoscute care pot duce la duplicarea unor
cereri din partea clientului.

In cazuri rare, retransmiterea unui SYN cu TFO si date in payload poate duce la creearea unei conexiuni duplicate la server.
Acest lucru poate fi dezastruos daca, de exemplu, serverul se ocupa de tranzactii financiare si proceseaza o tranzactie de doua ori.

In TLS 1.3, un client care a comunicat in prealabil cu un server poate sa deschida 

% Both TCP Fast Open and 0-RTT TLS Session Resumption have well-known issues that could lead to replays.
% Under rare circumstances, a resent TFO SYN could lead to the establishment of a duplicate connection
% at the server. This can be disastrous if, for example, the server is handling financial transactions
% and erroneously performs a transaction twice.
% As for TLS, the initial chunk of data sent out by the client during 0-RTT Session Resumption can be replayed
% % across multiple connections
% by an on-path attacker, or even unwittingly by the client itself, if it is using TFO.
% 
% In order to safely use TFO and 0-RTT Session Resumption on the client-proxy leg, 
% % Using SOCKS v6 in conjunction with either of them transfers the risks to the application-layer protocol
% % that is being proxied.
% % Some protocols can not tolerate these risks. In order to
% % still enjoy the benefits of either or both of said features,
% we have implemented a mechanism whereby SOCKS
% requests can be made idempotent\footnote{TFO is used on the proxy-server leg only if the client explicitly requests it.}.
% The proxy will honor an individual
% request at most once, regardless of how many times it has been replayed.
% 
% To protect against duplicate SOCKS requests, authenticated clients can request, and then spend, idempotence \emph{tokens}. A token can only be spent on a single SOCKS request.
% 
% Tokens are 4-byte unsigned integers in a modular 4-byte space. Therefore, if \(x\) and \(y\) are tokens, \(x\) is less than \(y\) if \( 0 < (y - x) < 2^{31} \) in unsigned 32-bit arithmetic.
% 
% Proxies grant contiguous ranges of tokens called \emph{token windows}. Token windows are defined by their base (the first token in the range) and size.
% Windows are shifted (i. e. have their base increased, while retaining their size) by the proxy.
% 
% Well-behaved clients attempt to spend their tokens in order. This, however, does not guarantee that the requests will be received or processed in order by the proxy.
% Requests can also be dropped by the network.
% 
% The proxy tracks the status of the tokens in the window, always shifting it past spent tokens.
% To prevent low-order unspent tokens from stalling the window's advancement, they can be forfeited subject to a parameter called the \emph{high water mark}.
% % The forfeiture of unspent tokens is driven by a parameter called the high water mark (which is strictly smaller than the window size).
% If the highest spent token exceeds the high water mark, the window is shifted.
% Increasing this parameter makes the proxy more tolerant to request reordering.
% 
% Our chosen solution has the following properties:
% \emph{a) Light weight}: The memory usage is constant,
% the proxy performs only one memory allocation per client, and all operations are cheap.
% This mechanism does not open up an avenue for DoS by itself.
% \emph{b) Resilient to proxy crashes}: Restarting the proxy would ultimately result in a new token window being allocated.
% Since the new window is very unlikely to overlap with the old one, the odds of honoring an old request are very low.
% \emph{c) Resilient across time}: To successfully replay a request, an attacker would have to wait until the window
% shifts back to the same position (i. e until the client makes 4 billion requests).
% \emph{d) Timely}: Token window advertisements are sent immediately after receiving the request, as part of the authentication reply.
% \emph{e) Unrestrictive}: The only restriction placed on the client is that it can make a limited number of requests per client-proxy RTT,
% which is not an issue if the window size is large enough.
% Token expenditure requests and window advertisements can be received out-of order.

\section{Opțiuni de configurare a sockeților pe proxy}

